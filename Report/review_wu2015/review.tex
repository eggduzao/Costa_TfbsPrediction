%%%%%%%%%%%%%%%%%%%%%%%%%%%%%%%%%%%%%%%%%%%%%%%%%%%%%%%%%%%%%%%%%%%%%%%%%%%%%%%%%%%%%%%%%%%%%%%%%%%
% 
% Review Paper: Improving Computational Identification of Cooperative Transcription Factors in
%               Yeast Using TF-Gene Direct Regulation
% Journal: Bioinformatics
% Manuscript ID: BIOINF-2015-0010
% Category: Original Paper
% Submission Data: 05-Jan-2015
% Authors: Wu, Wei-Sheng and Lai, Fu-Jou
% 
%%%%%%%%%%%%%%%%%%%%%%%%%%%%%%%%%%%%%%%%%%%%%%%%%%%%%%%%%%%%%%%%%%%%%%%%%%%%%%%%%%%%%%%%%%%%%%%%%%%


\documentclass[11pt]{article}
\usepackage[body={6.25in,8.85in}]{geometry}
\usepackage{url}
\usepackage[scaled]{helvet}
\usepackage[authoryear,round,longnamesfirst]{natbib}

\begin{document}
\title{Paper Review: Improving Computational Identification of Cooperative Transcription Factors in Yeast Using TF-Gene Direct Regulation Data} \date{}

\author{Eduardo G. Gusm\~{a}o}

\thispagestyle{empty}

\maketitle

\setlength{\parskip}{0.5cm}

\section{Comments to the Editor}

The method presented on this study consists simply on using direct gene-TF regulation data coupled with traditional hypergeometric test to identify cooperative regulatory interactions between TFs. This approach is compared to 12 methods which are not very recent. The competing methods used ChIP-chip data instead of direct gene-TF data and a different range of metrics to calculate TF cooperativity. It seems quite obvious that using the direct available gene-TF regulation data would yield better results given the metrics in which the algorithms were evaluated. My main critique is that there is not enough novelty in the method, i.e. they simply shown that using a better data source yields better results.

Furthermore, for most organisms (in which such comprehensive gene-TF regulation data is not available) this method is not applicable. In fact, recent studies in humans rely on chromatin dynamic assays such as ChIP-seq, DNase-seq, MethylCap-Seq and MNase-seq; since the establishment of direct gene-TF regulatory data for each cell type and study conditions is difficult to achieve. Also, the number of studies that use ChIP-seq on Yeast are starting to increase. For these reasons, the relevance of the research is also poor.

Although the major point of the study is that a more biologically relevant approach is preferred, I would condition the acceptance of this research paper on the ability of the authors to address the major points stated below.

\section{General Comments}

In this study the authors use a novel approach to predict cooperative interaction between transcription factors in yeast. Previous studies focused on the usage of cell-specific transcription factor binding information tiling microarrays data (ChIP-chip). This study directly used TF-gene regulation data available on YEASTRACT and YTRP. Their algorithm performs the traditional two-step approach: (1) data supported by TF binding and TF regulation data was obtained and (2) the statistical significance of cooperative TF pairs is assessed using a hypergeometric test. Their method outperforms 12 competing algorithms that use ChIP-chip data. Finally, they claim that computational cooperative TF prediction can be improved by using a more biologically relevant approach.

\section{Specific Comments for Revision}

\subsection*{a. Major}

\noindent{\textit{\textbf{1.}}} My major concern is that the approach taken here seems to be very specific and not innovative. According to the document: (1) Data was taken from known TF-gene regulation interaction; (2) An already published interaction measure was used on such data and (3) A simple p-value cutoff was used to obtain the putative cooperative interaction predictions. It seems quite obvious that using direct (biologically verified) regulation data leads to better results than using ChIP-chip, in which certain assumptions regarding the actual regulatory role have to be made. In my understanding the ChIP-chip (or more recently ChIP-seq) approach is used not only to capture the cell-type-- or condition--specificity (which does not seem to be discussed in this paper) but also because direct gene-TF regulatory evidence is not easily obtained for all TFs. This is because the regulatory landscape changes, for each organism, due to many factors such as cell type, condition of the cell under study and response to stimuli. Given that, it seems that such approach can not be performed for nearly all other eukaryotic organisms, in which direct gene/TF regulatory evidence is lacking.

\noindent{\textit{\textbf{2.}}} It would be of great interest to observe the accuracy when all other competing algorithms are applied to the direct gene/TF regulation data. I agree that more biologically relevant information should always be used. Nevertheless, the significance of computational algorithms should not be assuaged, especially when methods are being applied to different input data.

\noindent{\textit{\textbf{3.}}} It was not clear to me how the $p$-value was selected for the 46 pairs reported as significant in the Supplementary Table~S1. The authors already mention that there are studies in which different ways were developed to determine optimal $p$-value thresholds. I understand that the proposed algorithm outperformed others when the $p$-value was varied. But it was not clear how an optimal $p$-value is established, given the proposed algorithm.

\noindent{\textit{\textbf{4.}}} The approach presented on this study only verifies cooperation pairs. It would be interesting to generalize the test to any number of K cooperative TFs.

\noindent{\textit{\textbf{5.}}} The performance index 1 is a specific case of performance index 2 (when the average between the reciprocal distances equals 1). I wonder whether these metrics should be generalized as a single one. This would be preferable unless direct interactions have greater importance than indirect interactions.

\noindent{\textit{\textbf{6.}}} Most motifs have DNA binding affinity for particular nucleotides. A large repository of Position Weight Matrices (PWMs) for yeast is available in~\cite{boer2012}. I believe the performance can be further improved by using motif binding information either by filtering impossible motif pairs or by being incorporated into the cooperativity metric.

\noindent{\textit{\textbf{7.}}} What is the execution time for the hypergeometric test p-value calculation? Since the algorithm is based on an exhaustive search I wonder how it would be applied to organisms with larger genomes. Probably a more sophisticated approach such as greedy search or dynamic programming should be considered.

\subsection*{b. Minor}

\noindent{\textit{\textbf{1.}}} The manuscript would be enriched by adding high impact publications (see~\cite{martinez2012}) to support claims made in the introduction and discussion of results.

\noindent{\textit{\textbf{2.}}} More sophisticated statistical tests could be performed on the analysis of Fig~2. In this analysis you have different methods and, in some cases such as index 1, many numerical results (values from each PCTFP). For example, a Friedman-Nemenyi test could be used to rank different methods.

\noindent{\textit{\textbf{3.}}} It is claimed that ``Computational identification of cooperative TFs is now a hot research topic'' (abstract and introduction). I would rather state that it has been an important research topic in the last years and recently is taking advantage of computational techniques.

\noindent{\textit{\textbf{4.}}} I would also briefly discuss non-cooperative interactions~\cite{giorgetti2010,swami2010}. Do these interactions characterize noise to ChIP-chip data? How do they impact accuracy on both ChIP-chip--based methods and direct regulation evidence--based methods?

\noindent{\textit{\textbf{5.}}} I think the manuscript could be enhanced by explaining the overall idea of the methods in which ChIP-chip is used to identify cooperative TF interactions.

\noindent{\textit{\textbf{6.}}} Please review the text for grammar/clarity issues. Some examples:
\begin{itemize}
  \item Page 1, last sentence of abstract (motivation): ``(...) not necessary (...)'' to ``(...) not necessarily (...)''
  \item Page 2, Section 2.2.3: The sentence starting with ``Here we use the average (...)'' is confusing. Maybe replace ``(...) to evaluate the performance of an algorithm.'' by ``(...) to evaluate its performance''.
  \item Page 3, Section 4: ``(...) sophisticated algorithms which all used (...)'' to ``(...) sophisticated algorithms, all of which used (...)''
\end{itemize}

\bibliographystyle{natbib}
\bibliography{document}

\end{document}


