
\documentclass{AbstractTemplate}

\usepackage{graphicx}
\usepackage{epstopdf}

\title{Late Poster Abstract: HINT-BC -- HMM-based Identification of Transcription Factor Footprints on Bias-Corrected DNase-seq Data}

\author{Eduardo G. Gusm\~{a}o\,\affref{ref1}$^{,}$\affref{ref2}, Martin Zenke\,\affref{ref2} and Ivan G. Costa\,\affref{ref1}$^{,}$\affref{ref2}$^{,}$\affref{ref3}$^{,*}$}

\affiliation{
  \aff{ref1}
  	{IZKF Computational Biology Research Group, RWTH Aachen University Medical School, Aachen, Germany.}
  \aff{ref2}
  	{Department of Cell Biology, Institute of Biomedical Engineering, RWTH Aachen University Medical School, Aachen, Germany.}
  \aff{ref3}
  	{Aachen Institute for Advanced Study in Computational Engineering Science (AICES), RWTH Aachen University, Germany.}
  $^{*}$ ivan.costa@rwth-aachen.de
}

\begin{document}

\maketitle

Word count: 234 / 250

After the development of high-throughput DNase I footprinting technique (DNase-seq), many computational methods have been proposed to automatically detect transcription factor (TF) footprints using such data. However, recent research shows that DNase-seq presents an intrinsic cleavage bias regarding the fact that the DNase I enzyme is more likely to digest DNA around certain k-mers. We modified our method -- HINT (HMM identification of TF footprints) -- to include a cleavage bias-correction step to test whether the DNase I cleavage bias correction could improve computational TF footprint identification. We evaluated our novel approach, termed HINT-BC, together with seven competing methods in a comprehensive evaluation data set which comprises 83 ChIP-seq data sets. We observed that our bias correction strategy mitigated the cleavage bias by evaluating the correlation between the amount of bias and the area under the ROC curve (AUC) statistics for all TFs ($p$-value < $0.05$). Interestingly, we find that DNase-seq profiles indicate that the bias-corrected signal fits better the DNA sequence binding affinity of the TFs than the uncorrected signal. As expected, our method presents a significantly higher AUC than the competing methods and the uncorrected version of our method (Friedman-Nemenyi hypothesis test; $p$-value $<0.05$). These findings suggest that proper correction of DNase-seq cleavage bias mitigates the impact of such bias regarding the performance of computational footprinting. Our method is available as a command-line tool within the framework of the Regulatory Genomics Toolbox (RGT).

\end{document}
