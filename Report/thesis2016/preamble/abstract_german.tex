\chapter*{Abstrakt (Deutsch)}

Die Transkriptionsregulation beschreibt die zeitliche und r{\"a}umliche Expre{\ss}ion der Gene. Mit Hilfe der Identifikation von transregulatorischen Elementen, wie beispielsweise \linebreak Transkriptionsfaktorbindestellen, k{\"o}nnen regulatorische Netzwerke be{\ss}er verstanden werden. \linebreak Regulatorische Netzwerke beschreiben zellul{\"a}re Proze{\ss}e wie zum Beispiel die Zellentwicklung und das Entstehen von Krankheiten.

Beim herk{\"o}mmlichen rechnergest{\"u}tzten Ansatz zur Identifikation von \linebreak Transkriptionsfaktorbindestellen wird auf Sequenzierungsmethoden zur{\"u}ckgegriffen, um die DNA des Genoms nach Sequenzen mit unterschiedlichen Bindungsneigungen zu Transkriptionsfaktoren (TF) zu durchsuchen. Mit diesem Ansatz ist es jedoch nicht m{\"o}glich aktive Bindestellen \linebreak vorherzusagen. Eine aktive Bindestelle ist beispielsweise dann gegeben, wenn an der DNA-Sequenz ein TF bindet. Dieser auf Sequenzierungstechniken beruhende Ansatz nimmt keinen Bezug \linebreak darauf, da{\ss} der Zustand des Chromatins dynamisch zwischen offen (so da{\ss} ein TF binden kann) und geschlo{\ss}en (so da{\ss} kein TF binden kann) wechseln kann.

Mit Sequenzierungsmethoden der n{\"a}chsten Generation (next generation sequencing) kann offenes Chromatin genomweit identifiziert werden. Beispiele hierf{\"u}r sind die Kombination von Chromatin ImmunoPrecipitation (ChIP-seq) oder DNase I Verarbeitung (DNase-seq) mit der \linebreak Sequenzierungstechnik. Aktuelle Studien haben belegt, da{\ss} die Verwendung von ChIP-seq und DNase-seq zur Bestimmung von offenem Chromatin einen positiven Einflu{\ss} auf die Identifikation von aktiven TFBS haben. Dabei wird die Suche nach charakteristischen DNA-Sequenzen auf die Bereiche eingeschr{\"a}nkt, an denen das Chromatin offen ist und die TF somit in einer zellspezifischen Art binden k{\"o}nnen.

Wir f{\"u}hren zum ersten Mal in dieser Arbeit ein rechnergest{\"u}tztes Rahmenwerk ein, das DNase-seq und ChIP-seq Daten kombiniert, um aktive TFBS vorherzusagen. Wir haben beobachtet, da{\ss} es bei aktiven TFBS ein ausgepr{\"a}gtes Muster in DNase-seq und ChIP-seq Daten gibt. Unser \linebreak Rahmenwerk f{\"u}hrt zun{\"a}chst eine Normalisierung des Signals aus und sucht dann in den Daten nach diesen Mustern, den sogenannten Fu{\ss}abdr{\"u}cken. Dabei wird das Genom mit einem Hidden Markov Modell segmentiert. Unsere Methode mit dem Namen HINT (HMM-basierte Identifikation von TF Fu{\ss}abdr{\"u}cken) ist als {\glqq}rechnergest{\"u}tzte Fu{\ss}abdruck Methode{\grqq} kla{\ss}ifiziert. 

In unserer Evaluierung{\ss}tudie haben wir die vorhergesagten Fu{\ss}abdr{\"u}cke von HINT mit bereits validierten Fu{\ss}abdr{\"u}cken verglichen. Dabei haben wir Statistiken erzeugt, um unsere Methode mit anderen zu vergleichen. Unsere Experimente sind mit insgesamt 14 verglichenen Methoden und 233 TF die umfangreichsten. 

Zudem haben wir HINT erfolgreich bei zwei biologischen Studien angewandt, um regulatorische Elemente, die bei bestimmten biologischen Bedingungen vorkommen, zu identifizieren. HINT ist ein n{\"u}tzliches rechnergest{\"u}tztes Rahmenwerk f{\"u}r biologische Studien in der regulatorischen Genomik.


