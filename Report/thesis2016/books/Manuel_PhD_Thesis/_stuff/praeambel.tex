% \documentclass[
% 	pdftex,              	% PDFTex verwenden
% 	a4paper,             	% A4 Papier
% 	halfparskip,
% 	12pt               		% 10pt Schrift
% ]{article}




\renewcommand*\chapterheadstartvskip{\vspace*{-1cm}}
\setkomafont{sectioning}{\normalfont\bfseries}
\setkomafont{captionlabel}{\normalfont\bfseries} 
%\setkomafont{pagehead}{\normalfont\bfseries\tiny} % Kopfzeilenschrift
%\setkomafont{descriptionlabel}{\normalfont\bfseries}
\newcommand{\todo}[1]{ \textbf{(TODO: #1)}}
\newcommand{\gt}[1]{ \langle \text{#1}\rangle } % genotype: for example <AA>
\newcommand{\E}{\mbox{I\negthinspace E}} %erwartungswert
\newcommand{\s}[1]{\texttt{\large{#1}}} %fuer qgramm und sequenzen, hat gleichen buchstabenabstand
\newcommand{\co}[1]{\texttt{\large{#1}}} %fuer code, und wenn Programme genannt werden: R, Python

% \newcounter{fig}
% \setcounter{fig}{1}
% \renewcommand{\thefigure}{\arabic{fig}}
% 
% \newcounter{tab}
% \setcounter{tab}{1}
% \renewcommand{\thetable}{\arabic{tab}}

% Einstellungen der Seitenränder
\usepackage[left=3cm,right=2.5cm,top=1.5cm,bottom=1.5cm,includeheadfoot]{geometry}

% Paket für Übersetzungen ins Deutsche und Font/Input Encoding
%\usepackage[ngerman]{babel}	
%\usepackage[latin1]{inputenc} 
\usepackage[utf8]{inputenc} 
\usepackage[T1]{fontenc}


\usepackage{mathptmx}
\usepackage[scaled=.92]{helvet}
\usepackage{courier}


% Gestaltung der Chapter und Section Ueberschriten
\usepackage{titlesec}
% \renewcommand{\thechapter}{\Roman{chapter}}
\titleformat{\chapter}[display]
{\bfseries\Large}
{\vspace{-5ex}\filleft\MakeUppercase{\chaptertitlename} \Huge\thechapter}
{0ex}
{\titlerule
\vspace{1.5ex}%
\filright}
[\vspace{1.5ex}%
\titlerule\vspace{-3ex}]

\titleformat{\section}[display]
{\bfseries\Large}
{}
{-2ex}
{
\thesection \hspace{0.3cm}\filright } 
[\vspace{-1ex} \parbox{0.618\textwidth}{\titlerule}\vspace{-1ex}]





% \addtokomafont{chapter}{%
%   \renewcommand*{\raggedsection}{\raggedleft}}


% Quotes (Anführungszeichen)
%\usepackage[babel,german=quotes]{csquotes}

% Tabellen
%\usepackage{array}
%\usepackage{booktabs}
%\usepackage{tabularx}
%\usepackage{longtable}



% Mathematik
\usepackage{amsmath}
\usepackage{amssymb}
\usepackage{amsthm}
\usepackage{stmaryrd}
\usepackage{cancel}

% % Python listing setup

\usepackage{color}
\usepackage[procnames]{listings}
\usepackage{textcomp}
\usepackage{setspace}
\usepackage{palatino}
\renewcommand{\lstlistlistingname}{Code Listings}
\renewcommand{\lstlistingname}{Code Listing}
\definecolor{gray}{gray}{0.5}
\definecolor{green}{rgb}{0,0.5,0}
\definecolor{lightgreen}{rgb}{0,0.7,0}
\definecolor{purple}{rgb}{0.5,0,0.5}
\definecolor{darkred}{rgb}{0.5,0,0}
\lstnewenvironment{python}[1][]{
\lstset{
language=python,
basicstyle=\ttfamily\small\setstretch{1},
stringstyle=\color{green},
showstringspaces=false,
alsoletter={1234567890},
otherkeywords={\ , \}, \{},
keywordstyle=\color{blue},
emph={access,and,as,break,class,continue,def,del,elif,else,%
except,exec,finally,for,from,global,if,import,in,is,%
lambda,not,or,pass,print,raise,return,try,while,assert},
emphstyle=\color{orange}\bfseries,
emph={[2]self},
emphstyle=[2]\color{gray},
emph={[4]ArithmeticError,AssertionError,AttributeError,BaseException,%
DeprecationWarning,EOFError,Ellipsis,EnvironmentError,Exception,%
False,FloatingPointError,FutureWarning,GeneratorExit,IOError,%
ImportError,ImportWarning,IndentationError,IndexError,KeyError,%
KeyboardInterrupt,LookupError,MemoryError,NameError,None,%
NotImplemented,NotImplementedError,OSError,OverflowError,%
PendingDeprecationWarning,ReferenceError,RuntimeError,RuntimeWarning,%
StandardError,StopIteration,SyntaxError,SyntaxWarning,SystemError,%
SystemExit,TabError,True,TypeError,UnboundLocalError,UnicodeDecodeError,%
UnicodeEncodeError,UnicodeError,UnicodeTranslateError,UnicodeWarning,%
UserWarning,ValueError,Warning,ZeroDivisionError,abs,all,any,apply,%
basestring,bool,buffer,callable,chr,classmethod,cmp,coerce,compile,%
complex,copyright,credits,delattr,dict,dir,divmod,enumerate,eval,%
execfile,exit,file,filter,float,frozenset,getattr,globals,hasattr,%
hash,help,hex,id,input,int,intern,isinstance,issubclass,iter,len,%
license,list,locals,long,map,max,min,object,oct,open,ord,pow,property,%
quit,range,raw_input,reduce,reload,repr,reversed,round,set,setattr,%
slice,sorted,staticmethod,str,sum,super,tuple,type,unichr,unicode,%
vars,xrange,zip},
emphstyle=[4]\color{purple}\bfseries,
upquote=true,
morecomment=[s][\color{lightgreen}]{"""}{"""},
commentstyle=\color{red}\slshape,
literate={>>>}{\textbf{\textcolor{darkred}{>{>}>}}}3%
         {...}{{\textcolor{gray}{...}}}3,
procnamekeys={def,class},
procnamestyle=\color{blue}\textbf,
}}{}




%\theoremstyle{mydefinition}
\newtheorem{mydef}{Definition}[chapter]
\newtheorem{mylemma}[mydef]{Lemma}
\newtheorem{mycorollary}[mydef]{Corollary}
\newtheorem{bsp}[mydef]{Beispiel}
\newtheorem{theorem}[mydef]{Theorem}

\newcommand{\argmax}[1]{\underset{#1}{\operatorname{arg}\,\operatorname{max}}\;} % argmax
\newcommand{\argmaxu}[1]{\operatorname{\operatorname{arg}\,\operatorname{max}_{#1}}}
% \newcommand{\nuc}[1]{\texttt{#1}}
\newcommand{\nuc}[1]{#1}
\newcommand{\signalmatrix}[1]{\mathbf{#1}} % Signal matrix, N x 2
\newcommand{\seq}{\mathcal{A}} %sequenz symbol
\newcommand{\w}{w} %weite symbol
\newcommand{\drsp}{\mathrm{DRSP}} %symbol DRSP
\newcommand{\drspart}{\mathrm{DRS}-\textnormal{Partition}} %symbol DRS-Partition
\newcommand{\zB}{z.\,B. }
\newcommand{\dhe}{d.\,h. }
\newcommand{\mds}{\mathrm{MDS}}
\newcommand{\mdsfett}{\mathrm{\textbf{MDS}}}
\newcommand{\defgl}{\mathrel{\mathop{\raisebox{1pt}{\scriptsize$:$}}}=} %:= definieren
\newcommand{\rc}[1]{\overline{#1}} %reverse complement
\newcommand{\onO}[1]{\mathcal{O}\left(#1\right)}
\newcommand{\onOg}[1]{\mathcal{O} \Big(#1\Big)}
\newcommand{\onOgg}[1]{\mathcal{O} \big(#1\big)}
\newcommand{\onOl}[1]{\mathcal{O} \Bigg(#1\Bigg)}
\newcommand{\pvalue}{\text{p-value}}
\newcommand{\mcdf}{$\tilde F_T$}
\newcommand{\hmmstate}[1]{{\tt #1}}

% \usepackage{listings}

%%%%%%%%%%%% CODE!!!!!!! PYTHON

\usepackage[procnames]{listings}
\usepackage{chngcntr}
% \counterwithout{lstlisting}{chapter}

\usepackage{color}
% \usepackage[procnames]{listings}
\usepackage{textcomp}
\usepackage{setspace}
\usepackage{palatino}

\definecolor{orange}{rgb}{1,0.5,0} 
\definecolor{gray}{gray}{0.5}
\definecolor{green}{rgb}{0,0.5,0}
\definecolor{lightgreen}{rgb}{0,0.7,0}
\definecolor{purple}{rgb}{0.5,0,0.5}
\definecolor{darkred}{rgb}{0.5,0,0}

% \usepackage[]{algorithm2e}
\usepackage[chapter]{algorithm}
\usepackage{algorithmicx}
\usepackage{algpseudocode}

\renewcommand{\thealgorithm}{\arabic{chapter}.\arabic{algorithm}} 

% \lstset{language=Python,
%   basicstyle=\ttfamily,
%   frame=trBL,
%   tabsize=2,
% %   numbers=left,
%   captionpos=b,
%   numbersep=5pt,
%   showstringspaces=false,
% %   commentstyle=\rmfamily\itshape,
%   commentstyle=\color{red}\slshape,
%   morecomment=[s][\color{lightgreen}]{"""}{"""}
% }
%\usepackage{amsfonts}
%\usepackage{amsthm}

% Farben im PDF
\usepackage{color}
\usepackage{colortbl}
\definecolor{hellgrau}{rgb}{0.92,0.92,0.92}
\definecolor{dunkelgrau}{rgb}{0.8,0.8,0.8}


%\usepackage{xcolor}
\usepackage{enumerate}
\usepackage{enumitem}

% Textteile drehen 
\usepackage{rotating}

% Stellenweises Querformat
%\usepackage{lscape}

% ktenerweiterung
%\usepackage[multiple]{footmisc}

% Metainfomationen in PDFs 
\usepackage[pdftex]{hyperref}

% Zeilenabstand 1,5
% \usepackage{setspace}
% \onehalfspacing

% Bildunterschriften / Tabellenunterschriften
\usepackage[bf]{caption} 
\captionsetup{format=hang} 
\captionsetup{labelfont=it,textfont={it,footnotesize},margin=20pt, aboveskip=0pt}

\setlength{\abovecaptionskip}{0.2cm}


% Zeilenumbruch bei Bildbeschreibungen.
%\usepackage{setspace}

% PDF Datein einbinden
\usepackage{pdfpages}

% URL Packet
% \usepackage{url}
\PassOptionsToPackage{hyphens}{url}\usepackage{hyperref}

\usepackage{multirow}
\usepackage[round]{natbib} %uncomment if u do not want the cool cite method
%% Meine Einstellungen

%\setcounter{secnumdepth}{4}
%\setcounter{tocdepth}{4}

% Farbeinstellungen für die Links im PDF Dokument

\hypersetup{
	pdftitle={allhoff\_thesis},		% Titel des PDF Dokuments.
	pdfauthor={Manuel Allhoff},		% Autor des PDF Dokuments.
	pdfpagemode=UseOutlines,		% Inhaltsverzeichnis anzeigen beim öffnen
	pdfdisplaydoctitle=true,		% Dokumenttitel statt Dateiname anzeigen.
	%hyperfootnotes=false,
    colorlinks,						%
    citecolor=black,				%
    filecolor=black,				%
    linkcolor=black,				%
    urlcolor=black, 				%
	bookmarksnumbered=true,			% Überschriftsnummerierung im PDF Inhalt anzeigen.
	pdfstartview={Fit},
	pdfpagelayout=SinglePage
}

% Zeilenumbruch bei Bildbeschreibungen.
%\setcapindent{1em}
%\usepackage{setspace}

% Lorem Ipsum 
\usepackage{lipsum}

% Kopf- und Fuflzeile nach Fancy Header Style 
\usepackage{fancyhdr}
\pagestyle{fancy}
\fancyhf{}

% \rhead{\fancyplain{}{\slshape\leftmark}}
%Kopfzeile zentriert
\fancyhead[EL]{\nouppercase{\rightmark}}
\fancyhead[OR]{\nouppercase{\rightmark}}
%Linie oben
\renewcommand{\headrulewidth}{0pt}
%Fußzeile zentriert
\fancyfoot[EL]{\thepage}
\fancyfoot[OR]{\thepage}



%Linie unten
% \renewcommand{\footrulewidth}{0pt}

% Nur Kapitelname in Fußzeile anzeigen
%\renewcommand{\chaptermark}[1]{\markboth{#1}{}}
%\renewcommand*{\indexpagestyle}{plain}

% Numerierung von enumerate beginnt auf dem ersten Level mit Buchstaben
\renewcommand{\labelenumi}{(\alph{enumi})}
\renewcommand{\labelenumii}{(\roman{enumii})}

% Leerseite
\newcommand\leerseite{\newpage\thispagestyle{empty}\hspace{1cm}\newpage}


% Index erzeugen.
%\makeindex

