\chapter{Conclusion}
\label{chapter_conclusion}
This thesis contributes to the computational analysis of ChIP-seq data.
The main goal of this work was to develop algorithms that call differential peaks (DPs) in ChIP-seq data.

We propose the one-stage DPCs ODIN and THOR for the case without and with replicates in ChIP-seq experiments.
Both methods are based on a hidden Markov Model (HMM) to identify DPs, as HMMs segment the ChIP-seq signal by detecting peaks with variable size through the use of posterior decoding algorithms.
In the case without replicates, ODIN uses an HMM with a Binomial or Poisson distribution as emission.
The rationale is that the Binomial distribution models the number of successes in a sequence of independent Bernoulli experiments.
As we divide the genome into bins to construct the ChIP-seq signal, the reads can either fall into a particular genomic bin (true outcome) or into all other bins (false outcome).
Hence, the number of reads in a genomic bin is modelled by a Binomial distribution.
Furthermore, the Binomial distribution approximates the Poisson distribution for a high number of Bernoulli experiments, which is the case in our application.
Hence, we also evaluate the Poisson distribution for ODIN.
In the case with replicates, THOR uses a Negative Binomial distribution as emission to handle overdispersion.
We choose the Negative Binomial as it can be shown that it is equivalent to the Poisson distribution, where the mean is separately drawn from a Gamma distribution.
Furthermore, modelling overdispersion, which ChIP-seq data typically exhibits, is crucial for an accurate DP calling process.
As there is no analytical solution for the Baum-Welch algorithm with a Negative Binomial distribution, we estimate the corresponding parameters based on an empirically evaluated mean-variance function.
With regard to the HMM emission, we also propose a $p$-value estimation strategy to identify significant DPs.

A crucial aspect for the differential peak calling problem is the normalization, as ChIP-seq profiles typically exhibit different sequencing depths as well as different signal-to-noise ratios.
The widely used TMM normalization approach, which was originally developed for gene expression analysis, is based on the assumption that the number of reads in most genomic regions does not change across the conditions. 
This is not necessarily the case for protein interactions, as two distinct cells can have distinct amounts of proteins or histone modifications bound to their DNA. 
Particularly problematic with TMM is the effect of replicate-specific background noise. 
Hence, we propose the use of control regions for normalization.
The rationale is that control regions exhibit a high signal-to-noise ratio.
We propose housekeeping genes as control regions for the analysis of active histone marks.
For other proteins of interest, ChIP-PCR measurements can provide regions which can serve as control regions.

Several pre- and postprocessing steps are necessary to call DPs in ChIP-seq data: read filtering, fragment size estimation, GC-content normalization, control DNA normalization, sample normalization and artefact filtering.
ODIN and THOR perform all required steps which makes them the most complete methods for solving the differential peak calling problem.

Evaluation of DPCs is still an open problem in the research community.
To our best knowledge, we perform the most comprehensive evaluation study available.
First, we use biological data sets for the evaluation.
In the case without replicates, we propose the DAGE and, in the case with replicates, the DCA metric.
Both metrics are based on the idea of associating changes in protein-DNA bindings with changes in gene expression of the same cellular condition.
For the case without replicates, we use 15 data sets to quantify 5 competing methods (MACS2, ChIPDiff, DESeq, MAnorm and DBChIP); and for the case with replicates, we use 12 data sets to evaluate 7 competing methods (DiffReps, DiffBind, MACS2, csaw, DESeq-IDR, DESeq-JAMM and PePr).
To ensure not being biased in our evaluation study, we use data sets from different kinds of proteins (TFs and activating histone marks), which results in different peak sizes in the signal.
Second, we propose an algorithm to simulate ChIP-seq reads of two biological conditions with potential replicates that contain DPs.
The simulation of ChIP-seq profiles can produce artificial and customized gold standards which can be used to extensively evaluate DPCs with regard to distinct data characteristics.

ODIN significantly outperforms all competing methods on the simulated data, where we vary the size of the protein domains as well as the size of peaks within the domain.
With regard to the DAGE metric, the performance of MACS2, DBChIP and DESeq is worse than ODIN's independent from the protein of interest (TF or histone modifications).
This emphasizes ODIN's flexibility which is caused by the use of the HMM for the signal segmentation step.
MACS2 predicts simulated DPs well for the easy scenario with high peak sizes, but shows poor performance for peaks with a high variance in their size.
MAnorm in combination with the SPC MACS also provides good results for the simulated as well as biological data which can be explained by its sophisticated normalization strategy.
ChIPDiff also uses an HMM to identify DPs.
However, it lacks to apply the pre- and postprocessing steps resulting in a poor overall performance.

THOR outperforms competing methods for most simulated and biological data sets.  
The difference in performance between THOR and its closest competing method is relatively better for data with high overdispersion and low quality. 
Moreover, THOR with the housekeeping gene normalization approach is the top ranked method for the biological data. 
In particular, THOR performs best for experimental conditions from the follicular lymphoma study. 
This study has overall lowest quality statistics (FRiP) and highest within condition variance scores (overdispersion). 
Indeed, THOR's framework includes the estimation of overdispersion quality measures, which can be used to guide the choice of normalization strategy. 

One competing method with an overall good performance is MACS2 (unpublished), which was ranked third on simulated data and second on biological data. 
Although there is no current description of MACS2, it is based on  the framework of the widely used SPC MACS~\cite{zhang2008}.
The performance of other tools varied across distinct experiments. 
While DESeq-IDR performed well on simulated data cases with low within condition variance and low number of replicates, it failed to call peaks on data with large variance. 
This is expected as IDR was conceived for a conservative peak detection on technical replicates. 
JAMM (with DESeq) had good performance on simulated data and is the only framework performing integrative analysis of single signal peak calling problems with replicates. 
Some methods, such as PePr and DiffReps, had a tendency to call peaks larger than other tools and the observed histone changes. 
This explains the average performance of these methods in our evaluation. 
Poisson-THOR, which can be seen as a version of ODIN supporting replicates with a distribution not coping with overdispersion, has poor results in most evaluated scenarios. 
This reinforces the importance of support to overdispersion on the presence of replicates.

Altogether, our evaluation studies show that ODIN and THOR are the best performing methods.
The performance is justified from their methodological aspects, as both tools use an HMM to intrinsically analyses windows of varying size during the detection of DPs. 
Other competing methods are based on fixed window searches (PePr, DiffReps, csaw) or predefined peaks (DESeq-JAMM, DESeq-IDR, DiffBind, MAnorm, DBChIP).

We also demonstrate THOR's power to call DPs by applying it to two use cases.
First, we show that THOR calls DPs that support rSNPs.
We call SNPs for the leukemia study (LYMP-FL-CC in Table~\ref{table_datasets_with_replicates}) and check whether they fall into DPs called by THOR.
Among all methods, THOR provides DPs with the highest amount of covered SNPs.
We show that SNPs influence TFBS which are associated with leukemia.
Second, we assign DPs called by THOR to genes for cDC and pDC peaks compared to CDP for H3K4me1.
As expected, the resulting genes are associated with dendritic cells.
Both use cases give reasonable findings which emphasize THOR's usefulness in the ChIP-seq analysis.

\section{Future Work}
% future work
% DNA seq application possible?

A natural extension of ODIN and THOR is to consider more than 2 biological conditions which can yield more intricate epigenetic findings in biological and medical research.
For example, we could evaluate time series data as it is provided by \cite{kaikonnen2013}.
Also, we could analyse developmental series as described in the dendritic cells study.
Here, the MPP differentiate to CDP, and CDP differentiate to cDC or pDC cells.

Moreover, ODIN and THOR use an HMM to segment the ChIP-seq signal.
We restrict our analysis to the case of small or medium sized peaks.
However, HMMs are flexible, such that our methods could also call peaks in large protein domains.
A systematic evaluation of this idea could potentially offer a wide range of further applications for ODIN and THOR.

Furthermore, calling DPs in signals describing distinct biological conditions is not restricted to the ChIP-seq technique.
A novel sequencing application is SHAPE-seq~\citep{Lucks2011, Loughrey2014} which investigates the RNA structure.
For SHAPE-seq, solving the differential peak calling problem is potentially as import as for ChIP-seq, since comparing the RNA structure of two biological conditions leads to a deeper understanding of the underlying biological mechanisms in cells.
% Our methods can also be helpful to detect copy number variations (CNVs).
% CNVs describe the differences between a sample and a reference genome in the number of copies of genomic regions.
% These difference typically results in different read counts, which can be described as DPs,  for the genomic regions.

The idea of DCA and DAGE is that gene expression correlates to certain histone modifications of the same cellular conditions.
However, this approach can oversimplify the problem.
~\cite{Maze2014} state that histone modification levels can exhibit opposite changes with regard to the gene expression among a gene.
Moreover, the interplay between certain proteins and histone modifications may lead to changes in the histone modification level. 
For example, while the histone modification H3K4me3 is usually enriched at active gene promoters and therefore associated to transcription, its level can also decrease at certain genes when RNA polymerase II interacts with the gene.
Hence, the working assumption for our metrics is that in the majority of cases certain histone modification are consistently associated to gene expression changes in the same biological condition.
Furthermore, it is possible to generalize this idea of DCA and DAGE.
Instead of only considering gene expression data, DPs that are based on active histone modifications can be associated to other active histone modification that are known to be featured in the same biological conditions.

Our normalization approach is based on housekeeping genes as control regions in the case of active histone marks.
The rationale is that control regions on the one hand should have a similar signal across the conditions and on the other hand exhibit a low noise.
We could also evaluate the use of consensus peaks across the conditions as control regions.
Consensus peaks could be called by IDR or JAMM.
However, this approach includes the additional peak calling step in the normalization procedure.
Peak calling is error prone and its performance depends on the method's parametrization as well as the shape of the peaks in the ChIP-seq profiles.

% thesis ending
% In this thesis we propose methods to call DPs in ChIP-seq signals that are associated with distinct cellular conditions.
% Detecting DPs is an important and challenging task in many ChIP-seq based studies.
% Hopefully, our methods can make a small contribution to the research of living cells.

