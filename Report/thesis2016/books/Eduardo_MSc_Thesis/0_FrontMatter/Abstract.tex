
%%%%%%%%%%%%%%%%%%%%%%%%%%%%%%%%%%%%%%%%%%%%%%%%%%%%%%%%%%%%%%%%%%%%%%%%
% ----------------------------------------------------------------------
% Class: Abstract.tex
% Description: Contains the abstract.
% Author: Eduardo Gade Gusmao
% ----------------------------------------------------------------------
%%%%%%%%%%%%%%%%%%%%%%%%%%%%%%%%%%%%%%%%%%%%%%%%%%%%%%%%%%%%%%%%%%%%%%%%

\begin{abstracts}

% Background
The identification of cis-regulatory elements on DNA is crucial for the understanding of the complex regulatory networks that orchestrate diverse cell mechanisms such as differentiation, development and apoptosis. However, this task is very complex, given the great number of different transcription factors in the human genome. Currently, it is believed that there are around 1,500 factors, each of which can bind directly or indirectly to multiple loci. The standard computational approach for the detection of such regions consists in using Position Weight Matrices, which are probabilistic representations of the factor's binding affinities, to search the genome for regions likely to be binding sites. However, such approach results in a very high number of false positive hits, since it cannot distinguish between active / inactive binding sites and also because motifs are usually small and degenerate. To overcome these problems, recent techniques are being based on epigenetic features. The main idea is that some regions of the chromatin are densely packed in a closed structure, preventing the binding of regulatory proteins, while other regions are less packed (open chromatin), allowing such binding. Current research shows that data sources that are capable of signaling open regions, such as DNase~I digestion (obtained by DNase-seq) and histone modifications (obtained by ChIP-seq) can improve transcription factor binding sites prediction.

% Overview of the work
In this work, a continuous bivariate hidden Markov model is proposed which is capable of integrating epigenetic data sources, in order to evaluate if the results can be improved when compared to standard computational approaches or to single data source approaches. Besides that, a novel technique to estimate the parameters of the model was developed, making costly traditional procedures no longer necessary. It was observed that the proposed model significantly improves the sensitivity with low or no negative effect on the specificity when compared to open chromatin-only models.

{\bf Keywords: } Transcription Factor Binding Sites; DNase-seq; ChIP-seq; Histone Modifications; Hidden Markov Models.

\end{abstracts}