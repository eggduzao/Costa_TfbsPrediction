%%%%%%%%%%%%%%%%%%%%%%%%%%%%%%%%%%%%%%%%%%%%%%%%%%%%%%%%%%%%%%%%%%%%%
% Chapter 6 -> Conclusion
% Author: Eduardo G Gusmao
%%%%%%%%%%%%%%%%%%%%%%%%%%%%%%%%%%%%%%%%%%%%%%%%%%%%%%%%%%%%%%%%%%%%%
\chapter{Conclusion}
\label{cha:conclusion}

\graphicspath{{chapter6/figs/}}

% General statement
This work aimed at analyzing relevant features of computational footprinting methods, which use mathematical models to predict active transcription factor binding sites (TFBSs) with open chromatin data. We devised a novel computational footprinting framework -- HINT -- which uses DNase-seq and histone modification ChIP-seq data to predict active TFBSs. HINT is the first method to integrate the full resolution data of both DNase-seq and histone modification ChIP-seq. We performed a comprehensive evaluation of $14$ different computational footprinting methods, which showed that our method HINT significantly outperformed its competitors. Furthermore, we addressed a number of relevant characteristics on computational footprinting methods. Finally, we presented real case scenarios in which footprint predictions obtained with HINT aided in the understanding of regulatory mechanisms.

% HINT Model
\subsubsection{HINT Method}

% Topologies
We devised HINT -- a novel HMM-based computational footprinting method that segments the genome based on the full resolution signals of DNase-seq and histone modification ChIP-seq data. We investigated the performance of five different HMM topologies. The \textsc{original DNase + histone} topology was created to recognize the grammar of active TFBSs. Furthermore, we devised two topologies which also combine DNase and histone data: the \textsc{DNase + histone asymmetric peaks} topology considers the intrinsic asymmetry observed for histone modification peaks and the \textsc{DNase + histone without slope} topology consists of a simpler HMM model which considers only signal intensity. Moreover, we designed two additional topologies which: use only DNase-seq data and use only histone modification ChIP-seq data. We observed that the \textsc{original DNase + histone} HMM topology outperforms all other HMM topologies (Section~\ref{sec:ps.hmm.topology}). However, it is noticeable that the \textsc{DNase-only} topology's accuracies are very close to the \textsc{DNase + histone} topologies' accuracies. On the other hand, the \textsc{histone-only} topology presented the lowest accuracies. Our results showed that: (1) The proper integration of DNase-seq and histone modifications increases the accuracy of the prediction of active TFBSs and (2) The DNase-seq data has a great predictive power given its high spatial specificity.

% Histone modification
Furthermore, we tested, for the \textsc{original DNase + histone} topology, a number of different histone modification combinations. We tested models using individual, pairs and triples of the activating histone modifications H3K4me1, H3K4me3, H3K9ac, H3K27ac and H2A.Z. We showed in Section~\ref{sec:ps.combination.histone.modifications} that many combinations perform equally well. However, the histone modifications H3K4me1, H3K4me3 and H3K27ac seem to be particularly good predictors of open chromatin regions. As expected, models containing more histone modifications generally outperformed models with less histone modifications. However, the increase in performance is smaller when considering higher number of histone modifications. This result, together with the fact that one of the goals of our model is to generate accurate predictions with as few assays as possible, does justify an optimal number of assays between a combination of two to three histone modifications.

% Model training
With regard to HINT's training, we observed that it is cell-type train-independent (Section~\ref{sec:ps.hmm.training}). This means that an HMM model trained with data from one cell type does not present a significant change in accuracy when applied to another cell type. In practice, one could use a trained HMM (for a particular topology) and apply it to data from any other cell type. Although we did not test this claim on different organisms, this seems to be the case, since the training robustness stems from the efficacy of the normalization strategy. This result is very important from a practical perspective because it allows the use of the method without the need to train new HMM models, given new data.

% Guidelines for Computational Footprinting
\subsubsection{Guidelines for Computational Footprinting}

% Introduction
Until now, it was not clear the extent in which experimental issues such as the DNase-seq sequence cleavage bias and the transcription factor (TF) residence time had in the performance of computational footprinting methods. We performed an in-depth investigation of a number of features relevant for the identification of active TFBSs using open chromatin data. We highlighted three insightful experiments: (1) the selection of an optimal scoring strategy for computational footprinting methods and whether such scoring strategy alone could outperform more complex approaches; (2) the impact on performance of the DNase-seq sequence cleavage bias and (3) issues regarding the TF residence time.

% Footprint Scoring Strategy
The TC-Rank is a computational footprinting method which consists on scoring and ranking motif-predicted binding sites (MPBSs) based on the tag count (TC) metric. In contrast to positive evaluations of the TC-Rank by previous works~\citep{cuellar2012,he2014} we show that it has poor sensitivity performance as indicated by the area under the receiver operating characteristic (ROC) curve (AUC) at low false positive rate (FPR) levels. Such poor sensitivity was further evidenced by observing the very low FP-Exp values of the gene expression evaluation methodology. The ability of a footprint-specific metric, such as the FLR, to distinguish a change in binding events appears to be a distinct advantage of computational footprinting methods over a more general statistic, such as the TC, that only captures overall DNase hypersensitivity in a large window around MPBSs. On the other hand, as pointed in Section~\ref{sec:footprint.ranking.strategy} the TC metric outperformed the footprint score (FS), position weight matrix (PWM) bit-score and method-specific scoring metrics on ranking footprints. This shows that, while using TC to rank MPBSs and applying a cutoff strategy does not provide good results, using TC to rank already-predicted footprints is the best strategy observed.

% Correction of DNase-seq Sequence Cleavage Bias
The refined DNase-seq protocol and experimental artifacts presented in~\cite{he2014} underscore that robust \emph{in silico} techniques are required to correct for experimental artifacts and to derive valid biological predictions. In Section~\ref{sec:impact.dnase.sequence.cleavage.bias} we showed that the correction of DNase-seq signal using the ``DNase hypersensitivity site (DHS) sequence cleavage bias'' approach estimates virtually removes the effects of sequence bias artifacts on computational footprinting. We demonstrated that such correction can be performed prior to the execution of the computational footprinting method. On the other hand, ignoring experimental artifacts might lead to false predictions, as observed previously for Neph et al.'s predicted \emph{de novo} motifs~\citep{neph2012a,he2014}. 

% Transcription Factor Residence Time
It was shown in~\cite{sung2014} that TFs with low residence time do not present a recognizable footprint pattern. Therefore, these factors would not be accurately predicted by computational footprinting methods. This issue was discussed in details in Section~\ref{sec:impact.tf.residence.time}. Although the TF residence time is not an issue that can be solved computationally, we showed that we can use the protection score to indicate footprints of TFs with potential short binding time. Such footprint predictions of TFs with low protection score should be interpreted with caution.

% Comparative Analyses on Computational Footprinting Methods
\subsubsection{Comparative Analyses on Computational Footprinting Methods}

% Comparative 1
Our comparative evaluation analysis presented in Section~\ref{sec:computational.footprinting.methods.comparison} indicates the superior performance (in decreasing order) of HINT, DNase2TF and PIQ in the prediction of active TFBS in all evaluated scenarios. Moreover, tools implementing these methods were user-friendly and had lower computational demands than other evaluated methods. Clearly, the choice of computational footprinting approaches should also be based on experimental design aspects. For example, PIQ is the only method supporting analysis of replicates and time-series. On the other hand, studies requiring footprint predictions for latter \emph{de novo} motif analysis should use segmentation approaches as HINT or DNase2TF.

% Comparative 2
The availability, usability and scalability of software tools implementing the methods are also important features. Neph, HINT, PIQ and Wellington provide tutorials and software to run experiments with few command line calls. Of those, only HINT, PIQ and Wellington natively support standard genomic formats as input. Site-centric methods Cuellar, BinDNase, Centipede and FLR require a single execution and input data per TF and cell type, while segmentation methods require an execution per cell type only. These site-centric methods have computational demands $5$ times (FLR and Cuellar) to $50$ times (BinDNase and Centipede) higher than the slowest segmentation method (Wellington) in our comparative analysis using the {\tt Benchmarking Dataset} (Table~\ref{tab:comp.resource}).

% Comparative 3
Examples of the infeasibility of site-centric methods on the basis of processing time can also be taken from the case studies presented here (Section~\ref{sec:case.studies}). The segmentation approach HINT was executed four times in the dendritic cell case study (one time for each cell type) and one time in the HUVEC inflammation case study (only the cell type HUVEC was analyzed). The total running time of these five computational footprinting methods was \approxy$140$ hours (or \approxy$1.5$ hour in a $100$-core computational cluster). On the other hand, a site-centric approach would have to be executed for each TF in which we are interested in performing the TF enrichment analysis, for each cell type. This makes a total of \approxy$3000$ executions (given a restricted set of $600$ tested TFs), with an estimated execution time (based on the fastest site-centric method PIQ) of $579,000$ hours (or $241$ days in a $100$-core computational cluster).

% Conclusion 1
In conclusion, the assessment of computational footprinting methods is a demanding task, both computationally and technically. We have created a fair and reproducible benchmarking dataset for evaluation of TF binding using two validation approaches: using ChIP-seq and using gene expression. Although the rationales of the ChIP-seq and gene expression evaluation procedures are, in principle, very different, we observed a high agreement between their respective ranking of methods. This is evidence that this study provides a robust map of the accuracy of state-of-the-art computational footprinting methods. We provide all statistics, basic data and computational scripts to evaluate future computational footprinting methods. These resources are available at:

\begin{center}
\url{http://costalab.org/hint-bc}
\end{center}

% Conclusion 2
This is an important resource for increasing transparency and reproducibility of research on computational footprinting methods.

% Downstream Analyses and Case Studies
\subsubsection{Downstream Analyses and Case Studies}

% Intro
We present two common downstream analyses based on footprint predictions: the \emph{de novo} motif finding and the TF enrichment analysis.

% De Novo Motif Finding
We performed a \emph{de novo} motif finding procedure on footprints predicted with HINT combining the tools ``discriminative regular expression motif elicitation'' (DREME) and ``local motif enrichment analysis'' (CENTRIMO; Section~\ref{sec:denovo.motif.finding.footprints}). We identified six novel motifs associated to human embryonic cell type H1-hESC. Five of these motifs presented a particularly noticeable peak-dip-peak DNase-seq pattern, indicative of active TF binding. Although this analysis used a particularly simple experiment design, it exemplifies downstream analyses that can only be performed on footprint predictions from segmentation-based computational footprinting methods.

% Case studies
In Section~\ref{sec:case.studies} we presented two case studies in which our computational footprinting method HINT was successfully applied to identify TFs associated to different biological conditions. Both studies use the same downstream analysis on the predicted footprints: the TF enrichment analysis. We have shown that it is possible to explore different HINT's HMM topologies to address specific biological questions. The inclusion of such case studies had the main goal of showing the flexibility of our computational footprinting framework towards very different experimental scenarios. There were differences in the organism under study (mouse \emph{vs} human), in the availability of input data (histone modification ChIP-seq \emph{vs} DNase-seq) and in the biological questions asked. Nevertheless, in these two distinct scenarios, HINT's predictions aided in the identification of the respective key regulatory players.

%%%%%%%%%%%%%%%%%%%%%%%%%%%%%%%%%%%%%%%%%%%%%%%%%%%%%%%%%%%%%%%%%%%%%
% Section: Future Work
%%%%%%%%%%%%%%%%%%%%%%%%%%%%%%%%%%%%%%%%%%%%%%%%%%%%%%%%%%%%%%%%%%%%%
\section{Future Work}
\label{sec:future.work}

% Introduction
Although we covered a number of different challenges on the detection of active TFBSs with computational genomic footprinting methods, this research area still has some unexplored aspects. In this section we categorize these research opportunities as: computational footprinting method extension and further downstream analyses that can be performed with footprint predictions.

% Computational Footprinting Method Extension
\subsubsection{Computational Footprinting Method Extension}

% Other souces of bias
We have systematically investigated the DNase-seq sequence cleavage bias. However, as extensively explored in~\cite{meyer2014}, open chromatin genomic data are affected by other artifacts stemming from either the biological protocol or the computational pre-processing steps, such as: (1) chromatin fragmentation and size selection, (2) tissue-specific signal variability generated by the phenol chloroform extraction step commonly used to separate deoxyribonucleic acid (DNA) from protein, (3) DNA amplification biases and duplications, (4) particularities of read mapping algorithms and (5) TF binding characteristics. HINT can still be further expanded to encompass the correction of other experimental artifacts.

% Other data sources
Moreover, in this thesis we focused on using the open chromatin data from DNase-seq and histone modification ChIP-seq. However, there are novel experimental biological assays, such as ATAC-seq (assay for transposase-accessible chromatin)~\citep{buenrostro2013}, which are able to generate a nucleotide-resolution genome-wide map of open chromatin regions. ATAC-seq also exhibits active TF's footprint-like patterns similar to DNase-seq; and has two major advantages over DNase-seq: (1) ATAC-seq is less technical and (2) ATAC-seq requires a much lower number of cells to start the protocol. Furthermore, current efforts are being made in order to obtain the genome-wide signal for these experimental assays (DNase-seq, ChIP-seq and ATAC-seq) in a single-cell manner~\citep{buenrostro2015}. In this new paradigm, we are going to be able to study tissue heterogeneity by analyzing open chromatin profiles of individual cells.

% Further Downstream Analyses
\subsubsection{Further Downstream Analyses}

% Other analyses
Here we have shown two common downstream analysis: the \emph{de novo} motif finding (Section~\ref{sec:denovo.motif.finding.footprints}) and the TF enrichment analysis (Section~\ref{sec:case.studies}). Nevertheless, there are a number of different downstream analyses that can be performed on computationally-predicted footprints, such as: (1) integration with TF ChIP-seq data -- to determine the exact position where the TF is binding without relying on purely sequence-based metrics~\citep{pique2011}; (2) differential footprinting -- which searches for footprints that occur at particular cell conditions and finds, within these footprints, regulatory elements associated to such conditions~\citep{he2012}; and (3) integrative analyses -- in which the footprints are integrated with further chromatin dynamics information, such as the spatial configuration of the chromatin, to infer indirect binding events and protein tethering~\citep{thurman2012}.

% Improve current ones
Furthermore, no effort was made to improve current available downstream analysis, such as the \emph{de novo} motif finding, to handle the massive data generated by computational footprinting methods. The research of novel downstream methods which are devised particularly for footprints is needed to explore the full potential of computational footprint predictions.


