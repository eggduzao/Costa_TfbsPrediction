%%%%%%%%%%%%%%%%%%%%%%%%%%%%%%%%%%%%%%%%%%%%%%%%%%%%%%%%%%%%%%%%%%%%%
% Chapter 6 -> Concluding Remarks
% Author: Eduardo G Gusmao
%%%%%%%%%%%%%%%%%%%%%%%%%%%%%%%%%%%%%%%%%%%%%%%%%%%%%%%%%%%%%%%%%%%%%
\chapter{Concluding Remarks}
\label{cha:concluding.remarks}

\graphicspath{{chapter6/figs/}}

% Introduction
In this chapter we present a final discussion on the experimental results from our work (Section~\ref{sec:discussion}). Afterwards, insights are provided with regard to possible extensions of the work presented in this thesis (Section~\ref{sec:future.work}). Finally, we present this thesis' concluding remarks (Section~\ref{sec:conclusion.6})

%%%%%%%%%%%%%%%%%%%%%%%%%%%%%%%%%%%%%%%%%%%%%%%%%%%%%%%%%%%%%%%%%%%%%
% Section: Discussion
%%%%%%%%%%%%%%%%%%%%%%%%%%%%%%%%%%%%%%%%%%%%%%%%%%%%%%%%%%%%%%%%%%%%%
\section{Discussion}
\label{sec:discussion}

% HINT Parameter Selection
\subsubsection{HINT Parameter Selection}

% Model application
We observed that the basic DNase + histone HMM topology outperforms all other HMM topologies (Section~\ref{sec:ps.hmm.topology}). However, it is noticeable that the DNase-only topology's accuracies are very close to the DNase + histone topologies' accuracies, which shows the power of the DNase-seq data to predict active TFBSs. Furthermore, we tested the integration of a number of different histone modifications to perform footprint predictions together with DNase-seq. We showed in Section~\ref{sec:ps.combination.histone.modifications} that many combinations perform equally well. However, the histone modifications H3K4me1, H3K4me3 and H3K27ac seem to be particularly good predictors of open chromatin regions.

% Model training
With regard to HINT's training, we observed that it is cell-type train-independent (Section~\ref{sec:ps.hmm.training}). In practice, one could use a trained HMM (for a particular topology) and apply it to data from any other cell type. Although we did not test this claim on different organisms, this seems to be the case, since the training robustness stems from the efficacy of the normalization strategy.

% Footprint Scoring Strategy
\subsubsection{Footprint Scoring Strategy}

% TC
In contrast to positive evaluations of the TC-Rank by previous works~\cite{cuellar2012,he2014} we show that it has poor sensitivity performance as indicated by the AUC at low FPR levels. However, as pointed in Section~\ref{sec:footprint.ranking.strategy} the TC metric outperformed the footprint score, PWM bit-score and method-specific scoring metrics on ranking footprints.

% Correction of DNase I Sequence Cleavage Bias
\subsubsection{Correction of DNase I Sequence Cleavage Bias}

% Bias
The refined DNase-seq protocol and experimental artifacts presented in He et al.~\cite{he2014} underscore that robust \emph{in silico} techniques are required to correct for experimental artifacts and to derive valid biological predictions. In Section~\ref{sec:impact.dnase.sequence.cleavage.bias} we showed that the correction of DNase-seq signal with DHS sequence cleavage bias estimates virtually removes the effects of sequence bias artifacts on computational footprinting. We demonstrated that such correction can be performed prior to the execution of the computational footprinting method. On the other hand, ignoring experimental artifacts might lead to false predictions, as observed previously for Neph et al.'s predicted \emph{de novo} motifs~\cite{neph2012a,he2014}. 

% Comparative Analysis on Computational Footprinting Methods
\subsubsection{Comparative Analysis on Computational Footprinting Methods}

% Comparative 1
Our comparative evaluation analysis presented in Section~\ref{sec:computational.footprinting.methods.comparison} indicates the superior performance (in decreasing order) of HINT, DNase2TF and PIQ in the prediction of active transcription factor binding sites in all evaluated scenarios. Moreover, tools implementing these methods were user-friendly and had lower computational demands than other evaluated methods. Clearly, the choice of computational footprinting approaches should also be based on experimental design aspects. For example, PIQ is the only method supporting analysis of replicates and time-series. On the other hand, studies requiring footprint predictions for latter \emph{de novo} motif analysis should use segmentation approaches as HINT or DNase2TF.

% Comparative 2
The availability, usability and scalability of software tools implementing the methods are also important features. Neph, HINT, PIQ and Wellington provide tutorials and software to run experiments with few command line calls. Of those, only HINT, PIQ and Wellington natively support standard genomic formats as input. Site-centric methods Cuellar, BinDNase, Centipede and FLR require a single execution and input data per transcription factor and cell type, while segmentation methods require an execution per cell type only. These site-centric methods have computational demands $5$ times (FLR and Cuellar) to $50$ times (BinDNase and Centipede) higher than the slowest segmentation method (Wellington) in our comparative analysis using the {\tt Benchmarking Dataset} (see Table~\ref{tab:comp.resource}). Good examples of the infeasibility of site-centric methods on the basis of processing time are the case studies presented here (Section~\ref{sec:case.studies}). The segmentation approach HINT was executed four times in the dendritic cell case study (one time for each cell type) and one time in the Huvec inflammation case study (only the cell type Huvec was analyzed). The total running time of these five computational footprinting methods was \approxy$140$ hours (or \approxy$1.5$ hour in a $100$-core computational cluster). On the other hand, a site-centric approach would have to be executed for each transcription factor in which we are interested in performing the transcription factor enrichment analysis, for each cell type. This makes a total of \approxy$3000$ executions (given a restricted set of $600$ tested transcription factors), with an estimated execution time (based on the fastest site-centric method PIQ) of $579,000$ hours (or $241$ days in a $100$-core computational cluster).

% Conclusion
In conclusion, the assessment of computational footprinting methods is a demanding task, both computationally and technically. We have created a fair and reproducible benchmarking data set for evaluation of protein-DNA binding using two validation approaches: using ChIP-seq and using gene expression. Although the rationales of the ChIP-seq and gene expression evaluation procedures are, in principle, very different, we observed a high agreement between their respective ranking of methods. This is evidence that this study provides a robust map of the accuracy of state-of-the-art computational footprinting methods.

% Transcription Factor Residence Time
\subsubsection{Transcription Factor Residence Time}

% TF residence
The issue regarding transcription factor binding time presented in Sung et al.~\cite{sung2014} was discussed in details in Section~\ref{sec:impact.tf.residence.time}. We successfully showed that the simple protection score can indicate footprints of transcription factors with potential short binding time. Thus, footprint predictions of transcription factors with low protection score should be interpreted with caution.

% De Novo Motif Finding
\subsubsection{\emph{De Novo} Motif Finding}

% De novo
We performed a \emph{de novo} motif finding procedure on footprints predicted with HINT combining the tools DREME and CENTRIMO (Section~\ref{sec:denovo.motif.finding.footprints}). Nevertheless, we were able to identify six novel motifs associated to human embryonic cell type H1-hESC. Five of these motifs presented a particularly noticeable peak-dip-peak DNase-seq pattern, indicative of active transcription factor binding. Although this analysis used a particularly simple experiment design, it exemplifies downstream footprint analyses that can only be performed on predictions from segmentation-based computational footprinting methods

% Case Studies and Downstream Footprinting Analyses
\subsubsection{Case Studies and Downstream Footprinting Analyses}

% Case studies
In Section~\ref{sec:case.studies} we presented two case studies in which our computational footprinting method HINT was successfully applied to identify transcription factors associated to different biological conditions. Both studies use the same \emph{post hoc} analysis on the predicted footprints: the transcription factor enrichment analysis. We have shown that it is possible to explore different HINT's HMM topologies to address specific biological questions. The inclusion of such case studies had the main goal of showing the flexibility of our computational footprinting framework towards very different experimental scenarios. There were differences in the organism under study (mouse \emph{vs} human), in the availability of input data (histone modification ChIP-seq \emph{vs} DNase-seq) and in the biological questions asked.

%%%%%%%%%%%%%%%%%%%%%%%%%%%%%%%%%%%%%%%%%%%%%%%%%%%%%%%%%%%%%%%%%%%%%
% Section: Future Work
%%%%%%%%%%%%%%%%%%%%%%%%%%%%%%%%%%%%%%%%%%%%%%%%%%%%%%%%%%%%%%%%%%%%%
\section{Future Work}
\label{sec:future.work}

% Introduction
Although we covered a number of different current challenges on the detection of active transcription factor binding sites with computational genomic footprinting methods, this research area still has some unexplored opportunities. In this section we categorize these research opportunities as: computational footprinting method extension and further footprint downstream analyses.

% Computational Footprinting Method Extension
\subsubsection{Computational Footprinting Method Extension}

% Other souces of bias
We have systematically investigated the DNase I sequence cleavage bias. However, as extensively explored in Meyer et al.~\cite{meyer2014}, NGS-based genomic data are affected by other artifacts stemming from either the biological protocol or the computational pre-processing steps, such as: (1) chromatin fragmentation and size selection, (2) tissue-specific signal variability generated by the phenol chloroform extraction step commonly used to separate nucleic acids from protein, (3) DNA amplification biases and duplications, (4) particularities of read mapping algorithms and (5) TF binding characteristics. HINT can still be further expanded to encompass the correction of other experimental artifacts.

% Other data sources
Moreover, in this thesis we focused on using the open chromatin data from DNase-seq and histone modification ChIP-seq. However, there are novel experimental biological assays, such as ATAC-seq (assay for tansposase-accessible chromatin)~\cite{buenrostro2013}, which are able to generate a nucleotide-resolution genome-wide map of open chromatin regions. ATAC-seq also exhibits active transcription factor's footprint-like patterns similar to DNase-seq; and has two major advantages over DNase-seq: (1) ATAC-seq requires much less technical and (2) ATAC-seq requires a much lower number of cells to start the protocol. Furthermore, current efforts are being made in order to obtain the genome-wide signal for these experimental assays (DNase-seq, ChIP-seq and ATAC-seq) in a single-cell manner~\cite{buenrostro2015}. In this new paradigm, we are going to be able to study tissue heterogeneity by analyzing open chromatin profiles of an individual cell. The extension of HINT to encompass these novel assays is a straightforward follow-up study.

% Further Footprint Downstream Analyses
\subsubsection{Further Footprint Downstream Analyses}

% Other analyses
Here we have shown two common footprint \emph{post hoc} analysis: the \emph{de novo} motif finding (Section~\ref{sec:denovo.motif.finding.footprints}) and the transcription factor enrichment analysis (Section~\ref{sec:case.studies}). Nevertheless, there are a number of different \emph{post hoc} analyses that can be performed on computationally-predicted footprints, such as: (1) integration with transcription factor ChIP-seq data -- to determine the exact position where the transcription factor is binding without relying on purely sequence-based metrics~\cite{pique2011}; (2) differential footprinting -- which evaluates the footprints that occurs at particular cell conditions and finds, within these footprints, regulatory elements associated to such condition~\cite{he2012}; and (3) integrative analyses -- in which the footprints are integrated with further chromatin dynamics information, such as the spatial configuration of the chromatin, to infer indirect binding events and protein tethering~\cite{thurman2012}.

% Improve current ones
Furthermore, no effort was made to improve current available downstream analysis, such as the \emph{de novo} motif finding, to handle the massive data generated by computational footprinting methods. The research of novel downstream methods which are devised particularly for footprints is needed to explore the full potential of computational footprint predictions.

%%%%%%%%%%%%%%%%%%%%%%%%%%%%%%%%%%%%%%%%%%%%%%%%%%%%%%%%%%%%%%%%%%%%%
% Section: Conclusion
%%%%%%%%%%%%%%%%%%%%%%%%%%%%%%%%%%%%%%%%%%%%%%%%%%%%%%%%%%%%%%%%%%%%%
\section{Conclusion}
\label{sec:conclusion.6}

% Conclusion
This work aimed at analyzing various features of computational footprinting methods, which use mathematical models together with NGS-based data to predict active transcription factor binding sites. Next, we present a summary of the contributions we performed:

\begin{itemize}
  \item \textbf{Extensive literature review:} This work brought together virtually all state-of-the-art computational footprinting methods published in high-impact journals. An extensive literature description of different strategies on the detection of active binding sites was performed, which contributes as a resource for future research.
  \item \textbf{Signal treatment:} Novel DNase-seq and histone modification ChIP-seq signal treatment approaches were developed and formalized. Such treatment framework has proven to be robust and applicable to a wide range of different datasets.
  \item \textbf{Novel computational footprinting method:} We devised a novel computational footprinting method based on hidden Markov models. It has proven to work on the basis of an extensive evaluation process.
  \item \textbf{DNase-seq experimental bias correction:} We devised an approach to correct DNase-seq sequence cleavage bias. Our experiments have proven its efficiency on bias mitigation. Furthermore, we observed a significant increase in performance when using our bias correction strategy.
  \item \textbf{Empirical studies:} A number of empirical analyses were executed. These analyses evaluated features such as method's parameter selection, experimental bias correction, optimal footprint scoring strategy and TF binding residence time, with regard to the performance of computational footprinting methods.
  \item \textbf{Comprehensive computational footprinting method comparison:} A comprehensive comparison including: (1) our novel HMM-based approach HINT; (2) nine state-of-the-art computational footprinting methods in the literature and (3) four control approaches (TC-Rank, FS-Rank, PWM-Rank and signal filters).
  \item \textbf{Novel validation:} A novel evaluation approach based on gene expression was developed. The ranking of method's accuracies based on this novel evaluation methodology correlated significantly with the traditional ChIP-seq evaluation approach. By using two independent evaluation techniques we were able to assess the method's performance with a high degree of statistical significance.
  \item \textbf{Case studies:} We successfully applied our computational footprinting method in two different studies to identify regulatory elements involved in specific biological conditions. Furthermore, we have shown a common footprint downstream analysis -- the \emph{de novo} motif finding.
\end{itemize}

% Conclusion 2
Finally, this study provides all statistics, basic data and scripts to evaluate future computational footprinting methods. These resources are available at:

\begin{center}
\url{http://costalab.org/hint-bc}
\end{center}

% Conclusion 3
This is an important resource for increasing transparency and reproducibility of research on computational footprinting methods.


