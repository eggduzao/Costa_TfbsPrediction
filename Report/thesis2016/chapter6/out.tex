%%%%%%%%%%%%%%%%%%%%%%%%%%%%%%%%%%%%%%%%%%%%%%%%%%%%%%%%%%%%%%%%%%%%%
% Chapter 6 -> Concluding Remarks
% Author: Eduardo G Gusmao
%%%%%%%%%%%%%%%%%%%%%%%%%%%%%%%%%%%%%%%%%%%%%%%%%%%%%%%%%%%%%%%%%%%%%
\chapter{Concluding Remarks}
\label{cha:concluding.remarks}

\graphicspath{{chapter6/figs/}}

% Introduction
In this chapter we present a final discussion on the experiments and results performed in our work (Section~\ref{sec:discussion}). Afterwards, insigths are provided with regard to possible extensions of the work performed here (Section~\ref{sec:future.work}). Finally, we present the concluding remarks (Section~\ref{sec:conclusion.6})

%%%%%%%%%%%%%%%%%%%%%%%%%%%%%%%%%%%%%%%%%%%%%%%%%%%%%%%%%%%%%%%%%%%%%
% Section: Discussion
%%%%%%%%%%%%%%%%%%%%%%%%%%%%%%%%%%%%%%%%%%%%%%%%%%%%%%%%%%%%%%%%%%%%%
\section{Discussion}
\label{sec:discussion}

% What we have done
This work aimed at analyzing various features of computational footprinting methods, which use mathematical models together with NGS-based data to predict active transcription factor binding sites. Next, we present a summary of the contributions we performed:

\begin{itemize}
  \item \textbf{Extensive literature overview:} This work brought together virtually all computational footprinting methods published in high-impact journals. An extensive literature description of many different approaches on the detection of active binding sites was performed, which contributes as a resource for future research.
  \item \textbf{DNase-seq experimental bias correction:} Two approaches for correcting DNase-seq experimental biases were developed and formalized. Empirical studies proven their efficiency on bias mitigation.
  \item \textbf{Signal treatment:} Novel DNase-seq and histone modification ChIP-seq signal treatment approaches were developed and formalized. These treatments not only aided the further method application but also in the visualization of results and empirical tests.
  \item \textbf{Novel models:} A novel computational footprinting method was developed. It was proven to work with regard to an extensive validation procedure.
  \item \textbf{Novel validation:} A novel validation approach based on TF gene expression, instead of ChIP-seq for TFs, was developed. Such an approach correlated significantly with the standard literature validation approach.
  \item \textbf{Parameter refinements:} An extensive refinement analysis was performed on the parameterization of: methods (our and competing), signal treatment and evaluation approaches.
  \item \textbf{Empirical studies:} A number of empirical analyses were executed. These analysis evaluated features such as CG content, protein families and TF binding residence time, with regard to the performance of computational footprinting methods.
  \item \textbf{Comprehensive method comparison:} Finally, a comprehensive comparison including: (1) two novel approaches (HMM-based and filter-based); (2) ten state-of-the-art methods in the literature and (3) three control approaches (TC, FS and PWM bit-score). Also, we make available a number of benchmarking data sets.
\end{itemize}

% What the results showed us
Furthermore, we observed very interesting results. These results led us to key conclusions, which are summarized in the following paragraphs of this section.

% Overal method comparison
Our comparative evaluation analysis on multiple computational footprinting methods indicates the superior performance (in decreasing order) of HINT, DNase2TF and PIQ in the prediction of active TFBSs in all evaluated scenarios. Moreover, tools implementing these methods were user friendly and had lower computational demands than other evaluated methods. Clearly, the choice of computational footprinting approaches should also be based on experimental design aspects. For example, PIQ is the only method supporting analysis of replicates and time-series. On the other hand, studies requiring footprint predictions for latter de novo motif analysis should use segmentation-based approaches as HINT or DNase2TF. In contrast to positive evaluations of the TC-Rank method by previous works~\cite{}, we show that it has poor sensitivity performance as indicated by the AUC at low FPR levels. On the other hand, we showed that the TC  statistic is the best strategy to rank footprint predictions from other methods.

% Combination of DNase-seq and histone modification ChIP-seq improves TFBS detection
% TODO

% Hidden Markov Model Training is Cell-Independent
% TODO

% DNase-seq must be bias-corrected
The refined DNase-seq protocol and DNase I cleavage bias presented in He et al.~\cite{} underscore that robust in silico techniques are required to correct for experimental artifacts and to derive valid biological predictions. The correction of DNase-seq signal on an experiment-specific manner virtually removes the effects of the experimental and cleavage biases on computational footprinting. We demonstrated that such correction can be performed prior to the execution of the computational footprinting method. On the other hand, ignoring cleavage bias might lead to false predictions, as observed previously for predicted de-novo motifs <Sup. Fig. 10>.

% Models assuage the impact of further biases
% XXX such as CG content and different protein family binding properties

% Residence time is an issue
Our extended analysis of the TF binding residence time issue presented in Sung et al.~\cite{} showed that this is an important feature to be considered when interpreting footprint results. Moreover, the simple protection score can indicate footprints of TFs with potential short binding time, such as nuclear receptors, discriminating them from TFs which binds the DNA for a longer time, such as the insulator CTCF. Thus, footprint predictions of TFs with low protection score should be interpreted with caution.

% Case studies
Finally, we observed the utility and robustness of our computational footprint methods on two different case studies: on dendritic cells (histone-only HINT) and on Huvec cell lines (DNase+histone HINT). The regulatory landscape predicted by our method aided in the study of the differentiation of dendritic cells and on the understanding of NF-$\kappa$B co-binding partners during inflammatory response.

%%%%%%%%%%%%%%%%%%%%%%%%%%%%%%%%%%%%%%%%%%%%%%%%%%%%%%%%%%%%%%%%%%%%%
% Section: Future Work
%%%%%%%%%%%%%%%%%%%%%%%%%%%%%%%%%%%%%%%%%%%%%%%%%%%%%%%%%%%%%%%%%%%%%
\section{Future Work}
\label{sec:future.work}

% Introduction
Although our framework is validated with data from well-established repositories, there are still many research opportunities that can be categorized as: (1) method extensions; (2) further result processing and (3) application to biological research. In this section we will glance over these research opportunities.

% Method extensions
% TODO - ATAC-seq
% TODO - single cell
% TODO - integrative approaches

% Further result processing
% TODO - recognition of regulatory elements (de novo)
% TODO - classification of TFs

% Application to biological research
% TODO - describe the many areas that it can be applied
% TODO - finalize by saying that the integrative approach can be useful here


Experimental biases and limitations: Some experimental artifacts, such as DNase-seq cleavage bias, were addressed successfully6,17. However, computational approaches to process signals generated from DNase-seq, ATAC-seq and their single-cell versions still need to account for a number of complexities inherent from the experimental protocols15,18.

Recognition of cis-regulatory elements: Maps of cis-regulatory regions based on segmentation approaches were proven to be reliable. However, the identification of the TFs that bind to these binding regions is still an open research opportunity. Specially, the creation of a de novo motif finding algorithm designed to handle such big data and their intrinsic complexities. This novel algorithm would take advantage over competitors because it would be able to use not only DNA sequence information but also chromatin accessibility data.

Classification of TFs: It is currently available a number of ontologies and classification of TFs19. However, these approaches did not account for chromatin dynamics and the complexity which can be captured with NGS-based methods. It is very clear that the regulatory genomics field needs a standardized classification of cis- and trans-acting elements which accounts for the complexity discussed here.

Adaptation of current footprinting methods: There are currently no robust method to generate a regulatory map from technologies such as ATAC-seq or single-cell versions of DNase-seq and ATAC-seq. The knowledge used on similar footprinting methods can be adapted for these novel assays.

Chromatin biology context: Finally, with the proper tools, cell differentiation can be analyzed using an integrative approach in the context of chromatin biology.

%%%%%%%%%%%%%%%%%%%%%%%%%%%%%%%%%%%%%%%%%%%%%%%%%%%%%%%%%%%%%%%%%%%%%
% Section: Conclusion
%%%%%%%%%%%%%%%%%%%%%%%%%%%%%%%%%%%%%%%%%%%%%%%%%%%%%%%%%%%%%%%%%%%%%
\section{Conclusion}
\label{sec:conclusion.6}

% Conclusion
This work aimed at analyzing various features of computational footprinting methods, which use mathematical models together with NGS-based data to predict active transcription factor binding sites. The assessment of footprint methods is a demanding task, both computationally and technically. We have created a fair and reproducible benchmarking data set for evaluation of protein-DNA binding using two validation approaches: TF ChIP-seq based and FLR-Exp. Although the rationales of the ChIP-seq based and FLR-Exp evaluation procedures are, in principle, very different, we observed a high agreement between their respective ranking of methods. This is evidence that this study provides a robust map of the accuracy of state-of-the-art computational footprinting methods. Finally, this study provides all statistics, basic data and scripts to evaluate future computational footprinting methods. This is an important resource for increasing transparency and reproducibility of research on computational footprinting methods.


