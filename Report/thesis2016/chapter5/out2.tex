%%%%%%%%%%%%%%%%%%%%%%%%%%%%%%%%%%%%%%%%%%%%%%%%%%%%%%%%%%%%%%%%%%%%%
% Section: Discussion
%%%%%%%%%%%%%%%%%%%%%%%%%%%%%%%%%%%%%%%%%%%%%%%%%%%%%%%%%%%%%%%%%%%%%
\section{Discussion}
\label{sec:discussion.5}

% Introduction
In this chapter we provided an investigation on state-of-the-art computational footprinting methods. We show results regarding parameter selection, selection of optimal ranking criteria, biological experimental bias correction, issues regarding the transcription factor residence time, examples of post-hoc analyses such as \emph{de novo} motif finding and transcription factor enrichment. Furthermore, a comprehensive comparative study on a significant number of computational footprinting methods was performed.

% Bias
The refined DNase-seq protocol and experimental artifacts presented in He et al.~\cite{he2014} underscore that robust \emph{in silico} techniques are required to correct for experimental artifacts and to derive valid biological predictions. In Section~\ref{sec:second.footrank.biascorr} we showed that the correction of DNase-seq signal with DHS sequence cleavage bias estimates virtually removes the effects of sequence bias artifacts on computational footprinting. We demonstrated that such correction can be performed prior to the execution of the computational footprinting method. On the other hand, ignoring experimental artifacts might lead to false predictions, as observed previously for predicted \emph{de novo} motifs~\cite{neph2012a,he2014}. 

% Comparative 1
Our comparative evaluation analysis presented in Section~\ref{sec:computational.footprinting.methods.comparison} indicates the superior performance (in decreasing order) of HINT, DNase2TF and PIQ in the prediction of active transcription factor binding sites in all evaluated scenarios. Moreover, tools implementing these methods were user-friendly and had lower computational demands than other evaluated methods. Clearly, the choice of computational footprinting approaches should also be based on experimental design aspects. For example, PIQ is the only method supporting analysis of replicates and time-series. On the other hand, studies requiring footprint predictions for latter \emph{de novo} motif analysis should use segmentation approaches as HINT or DNase2TF. In contrast to positive evaluations of the TC-Rank by previous works~\cite{cuellar2012,he2014} we show that it has poor sensitivity performance as indicated by the AUC at low FPR levels. However, as pointed in Section~\ref{sec:second.footrank.biascorr} the TC metric outperformed the footprint score, PWM bit-score and method-specific scoring metrics on ranking footprints.

% Comparative 2
The availability, usability and scalability of software tools implementing the methods are also important features. Neph, HINT, PIQ and Wellington provide tutorials and software to run experiments with few command line calls. Of those, only HINT, PIQ and Wellington natively support standard genomic formats as input. Site-centric methods Cuellar, BinDNase, Centipede and FLR require a single execution and input data per transcription factor and cell type, while segmentation methods require an execution per cell type only. These site-centric methods have computational demands $5$ times (FLR and Cuellar) to $50$ times (BinDNase and Centipede) higher than the slowest segmentation method (Wellington) in our comparative analysis using the {\tt Benchmarking Dataset} (see Table~\ref{tab:comp.resource}). Good examples of the infeasibility of site-centric methods on the basis of processing time are the case studies presented here (Section~\ref{sec:case.studies}). The segmentation approach HINT was executed four times in the dendritic cell case study (one time for each cell type) and one time in the Huvec inflammation case study (only the cell type Huvec was analyzed). The total running time of these five computational footprinting methods was \approxy$140$ hours (or \approxy$1.5$ hour in a $100$-core computational cluster). On the other hand, a site-centric approach would have to be executed for each transcription factor in which we are interested in performing the transcription factor enrichment analysis, for each cell type. This makes a total of \approxy$3000$ executions (given a restricted set of $600$ tested transcription factors), with an estimated execution time (based on the fastest site-centric method PIQ) of $579,000$ hours (or $241$ days in a $100$-core computational cluster).

% TF residence
The issue regarding transcription factor binding time presented in Sung et al.~\cite{sung2014} was discussed in details in Section~\ref{sec:impact.tf.residence.time}. We successfully showed that the simple protection score can indicate footprints of transcription factors with potential short binding time. Thus, footprint predictions of transcription factors with low protection score should be interpreted with caution.

% Case studies
In Section~\ref{sec:case.studies} we presented two case studies in which our computational footprinting method HINT was successfully applied to identify transcription factors associated to different biological conditions. Both studies use the same \emph{post hoc} analysis on the predicted footprints: the transcription factor enrichment analysis. We have shown that it is possible to explore different HINT's HMM topologies to address specific biological questions. The inclusion of such case studies had the main goal of showing the flexibility of our computational footprinting framework towards very different experimental scenarios. There were differences in the organism under study (mouse \emph{vs} human), in the availability of input data (histone modification ChIP-seq \emph{vs} DNase-seq) and in the biological questions asked.

% Other types of analyses
Here we have shown two common footprint \emph{post hoc} analysis: the \emph{de novo} motif finding (Section~\ref{sec:denovo.motif.finding.footprints}) and the transcription factor enrichment analysis (Section~\ref{sec:case.studies}). Nevertheless, there are a number of different \emph{post hoc} analyses that can be performed on computationally-predicted footprints, such as: (1) integration with transcription factor ChIP-seq data -- to determine the exact position where the transcription factor is binding without relying on purely sequence-based metrics~\cite{pique2011}; (2) differential footprinting -- which evaluates the footprints that occurs at particular cell conditions and finds, within these footprints, regulatory elements associated to such condition~\cite{he2012}; and (3) integrative analyses -- in which the footprints are integrated with further chromatin dynamics information, such as the spatial configuration of the chromatin, to infer indirect binding events and protein tethering~\cite{thurman2012}.

% Conclusion
In conclusion, the assessment of computational footprinting methods is a demanding task, both computationally and technically. We have created a fair and reproducible benchmarking data set for evaluation of protein-DNA binding using two validation approaches: using ChIP-seq and using gene expression. Although the rationales of the ChIP-seq and gene expression evaluation procedures are, in principle, very different, we observed a high agreement between their respective ranking of methods. This is evidence that this study provides a robust map of the accuracy of state-of-the-art computational footprinting methods. Finally, this study provides all statistics, basic data and scripts to evaluate future computational footprinting methods. This is an important resource for increasing transparency and reproducibility of research on computational methods for DNase-seq data.


