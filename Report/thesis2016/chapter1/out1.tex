%%%%%%%%%%%%%%%%%%%%%%%%%%%%%%%%%%%%%%%%%%%%%%%%%%%%%%%%%%%%%%%%%%%%%%%%%%%%%%%%%%%%%%%%%%%%%%%%%%%
% Chapter 1 -> Introduction
% Author: Eduardo G Gusmao
%%%%%%%%%%%%%%%%%%%%%%%%%%%%%%%%%%%%%%%%%%%%%%%%%%%%%%%%%%%%%%%%%%%%%%%%%%%%%%%%%%%%%%%%%%%%%%%%%%%
\chapter{Introduction}
\label{cha:introduction}

\graphicspath{{chapter1/figs/}}

% This chapter
In this introductory chapter we present a motivation for the problem we are going to address in this thesis -- the computational identification of active transcription factor binding sites (Section~\ref{sec:problem.motivation}). The goal of this section is present the importance of this study. We also provide a quick glance over the contributions of our work (Section~\ref{sec:contributions}) and the structure of this thesis (Section~\ref{sec:document.structure}).

%%%%%%%%%%%%%%%%%%%%%%%%%%%%%%%%%%%%%%%%%%%%%%%%%%%%%%%%%%%%%%%%%%%%%
% Section: Problem Motivation
%%%%%%%%%%%%%%%%%%%%%%%%%%%%%%%%%%%%%%%%%%%%%%%%%%%%%%%%%%%%%%%%%%%%%
\section{Problem Motivation}
\label{sec:problem.motivation}

% Human genome project
In October $1990$ the human genome project started, with the goal of sequencing the complete human genome. From $1990$ to nowadays, the sequencing techniques have evolved very quickly. As an example, in $2000$ the cost to sequence a DNa fragment with length of one million nucleotides was approximately $\$5,300$, i.e. $\$95,300,000$ per genome. However, in $2014$ the total cost to sequence the complete human genome (\approxy$3.1$ billion nucleotides) finally reached the milestone of $\$1,000$~\cite{hayden2014}.

% Genome is not enough
A couple of years ago, it was believed that, in possession of the complete genome for a given organism, it would be possible to exactely determine its phenotype and disease succeptibility. However, after the analysis of the first genomes, it was clear that the simple determination of an organism's nucleotide sequence is not enough to explain the great diversity of biological processes. Such processes are governed by a complex chain of events involving regulatory mechanisms, in which genes are turned on and off in a dynamic manner.

% Gene regulation
These regulatory mechanisms drive the correct execution of biological processes such as development, proliferation, aging, differentiation and many others; and require a set of carefully orchestrated steps that depends on the correct spatial and temporal expression of genes~\cite{maston2006}. Therefore, the deregulation of gene expression is often linked to diseases~\cite{encode2012}. In the so-called post-genomic era, attention is turning to the understanding of how protein-coding genes (about $25,000$ in humans) and their products work, especially on their spatial and temporal expression patterns, which are established both at the cellular level and considering the body as a whole\cite{maston2006}.

% Regulatory elements
To understand the molecular mechanisms that dictate the cell's expression patterns, it is important to identify the regulatory elements involved in these activities. One of the most important regulatory features are transcription factors (TFs) -- proteins that bind on the DNA enhancing or repressing the expression of genes. These proteins bind to particular genomic regions called transcription factor binding sites (TFBSs). However, the identification of TFBSs is a dauting task for several reasons, such as: (1) there are about $1,500$ different TFs in the human genome, which bind to several TFBSs using different binding mechanisms; (2) although many TFs have affinity towards specific DNA sequences, these sequences vary considerably and have a very low number of conserved positions; and (3) the presence of a putative TFBS does not guarantee that a protein will be bound. Different cell types, cell conditions and stimuli response presents a different map of active TFBSs, i.e. TFBSs that are actually being bound by TFs.

% First methods to identify TFBSs
Experimental biological assays such as DNase I footprinting and chromatin immunoprecipitation are able to identify TFBSs with a very good accuracy. However, these methods are only capable of analyzing small genomic regions, which makes the genome-wide application of these methods virtually impossible. More recently, computational tools that are able to search for TFBSs genome-wide using information on the TFs' binding affinity on DNA sequence. The standard computational method following this approach is called motif matching. However, this approach also has its problems. The main one stems from the fact that it is not able to detect active TFBSs, since it uses only the DNA sequence information, which is the same in every cell of a particular organism.

% Current approach
The advances in sequencing technologies -- termed next-generation sequencing (NGS) -- mentioned previously have made possible to revisit the experimental low-throughput DNase I footprinting and chromatin immunoprecipitation techniques and make the necessary changes in order for them to be able to be applied in a genome-wide manner. Examples of such novel assays are the DNase-seq (DNase I footprinting followed by massively parallel sequencing) and ChIP-seq (chromatin immunoprecipitation followed by massively parallel sequencing). However, novel computational approaches are required to analyze the complex data generated by these assays.

%%%%%%%%%%%%%%%%%%%%%%%%%%%%%%%%%%%%%%%%%%%%%%%%%%%%%%%%%%%%%%%%%%%%%
% Section: Contributions
%%%%%%%%%%%%%%%%%%%%%%%%%%%%%%%%%%%%%%%%%%%%%%%%%%%%%%%%%%%%%%%%%%%%%
\section{Contributions}
\label{sec:contributions}

% Framework to analyze footprints
The main contribution of this work is the development of a novel computational framework to treat data generated with the DNase-seq and ChIP-seq technologies and a novel computational method to detect active transcription factor binding sites based on these data. We developed two novel active TFBS detection approaches: based on hidden Markov models and based on signal filtering. We performed a number of empirical experiments to increase the models' performance while avoiding data overfitting.

% Extensive literature overview and method comparison
Furthermore, we performed a thorough literature review and executed ours and competing methods in multiple data sets available from public repositories. We evaluated the methods using the standard literature approach based on ChIP-seq for TFs. Furthermore, we developed a novel evaluation methodology which includes gene expression instead of TF ChIP-seq, avoiding potential biases introduced by the latter. Comprehensive statistical analyses were performed to compare: two novel TFBS identification approaches, ten competing methods and three baseline approaches.

% Exploring features such as cleavage bias and TF binding time
Finally, we performed multiple empirical analysis that provided insights on the task of identification of active TFBSs using NGS-based data. Among these empirical analysis are: (1) computational insights into the models used in this work; (2) statistical analysis of the data; (3) parameterization of data treatment techniques, methods and evaluation approaches; and (4) the impact of multiple genomic features on the accuracy of the computational methods.

%%%%%%%%%%%%%%%%%%%%%%%%%%%%%%%%%%%%%%%%%%%%%%%%%%%%%%%%%%%%%%%%%%%%%
% Section: Document Structure
%%%%%%%%%%%%%%%%%%%%%%%%%%%%%%%%%%%%%%%%%%%%%%%%%%%%%%%%%%%%%%%%%%%%%
\section{Document Structure}
\label{sec:document.structure}

% Chapter 2
In Chapter~\ref{cha:background} we will introduce all the concepts needed for the understanding of our work. We will also discuss in detail the DNase-seq and ChIP-seq techniques. The problem we are addressing will be clearly stated and a comprehensive literature review will be performed.

% Chapter 3
In Chapter~\ref{cha:methods} we will formalize all methods used in our analyses. We will describe how we treated the input genomic signals and the two novel approaches introduced by our research: the HMM-based approach and the signal filtering-based approach.

% Chapter 4
In Chapter~\ref{cha:experiments} we will describe the full experiment design of this project. We will start by stating all the data sets and repositories. Then we will describe the standard literature evaluation methodology (based on ChIP-seq) and a novel evaluation methodology proposed here (based on gene expression). Finally, we will describe the execution of all competing methods analyzed in this research.

% Chapter 5
In Chapter~\ref{cha:parameter.refinements} we will describe all empirical analysis performed regarding the parameter tuning of: methods, signal processing and validation methodology. Our goal is to give precise answers on how specific parameters were selected.

% Chapter 6
In Chapter~\ref{cha:results} all results will be presented. These results include: the accuracies and statistical comparison of all methods under all evaluation methodologies, statistical analyses which led to meaningful insights within the data, empirical analyses on the computational aspects of our methodology and empirical analyses on the performance of the methods \emph{versus} multiple genomic features such as CG content and TF binding residence time.

% Chapter 7
In Chapter~\ref{cha:case.studies} we will show examples of the application of our tool in real biological research scenarios. We will present two case studies regarding the application of our methodology in dendritic cells~\cite{xxx} and in the Huvec cell line~\cite{xxx}.

% Chapter 8
In Chapter~\ref{cha:concluding.remarks} we will summarize all contributions performed by this research project. Moreover, we will highlighting all the key findings. Furthermore, we will discuss the future research opportunities which follows from this project. Finally, we will make our concluding remarks.

% Appendix
Further results which are too big to be placed in the main text can be found in Appendix~XXX. 


